\documentclass[11pt]{article}
\usepackage[margin=1in]{geometry}
\usepackage{graphicx}
\usepackage{hyperref}
\usepackage{amsmath,amssymb}
\usepackage{enumitem}
\usepackage{xcolor}
\usepackage{booktabs}
\usepackage{subcaption}
\usepackage[font=small]{caption}

\setlength{\parskip}{4pt}
\setlength{\parindent}{0pt}

\begin{document}

\section*{Weekly Progress Report, Feb 27 2026}

\subsection*{Literature Review}

This week I read the Kevrekidis paper and the Reiss paper:

\begin{enumerate}[leftmargin=*,itemsep=2pt]
  \item \textbf{Sonday, Singer \& Kevrekidis (2011),}
    arXiv:1104.0725.
    They build a Hermitian matrix from pairwise snapshot alignments
    and extract optimal shifts from its leading eigenvectors, then
    combine symmetry quotienting with dimensionality reduction via
    vector diffusion maps.

  \item \textbf{Reiss, Schulze, Sesterhenn \& Mehrmann (2018),}
    arXiv:1512.01985.
    They formally define sPOD for multiple transport velocities,
    introducing time-dependent shifts per transport component and
    optimizing them jointly via an iterative procedure.
\end{enumerate}

\textbf{Takeaway:} Our approach is simpler: single reference field
(temporal mean), FFT cross-correlation, one-shot, no iterative
optimization.  Works because our system has one dominant transport
mode.

\subsection*{What Our sPOD Does}

We convert particles on a periodic 2D box into $64\times64$ KDE
density images.  The flock translates, so standard POD wastes modes on
position.  We shift every frame to a common location via FFT
cross-correlation against the temporal mean, using
\texttt{numpy.roll} (exact, no interpolation).  After quotienting out
translation, POD only represents shape dynamics, and the singular
values decay much faster.

\subsection*{Experiments and Results}

I ran DYN1--DYN6 covering: gentle Vicsek, hypervelocity, high noise,
strong attraction, log/sqrt density transforms, and variable speed.
I also ran an ablation suite (XABL1--XABL8) as a $2^3$ factorial over
\{transform, simplex, spectral scaling\}, all with alignment on.
Reconstruction R$^2$ from the DYN suite (all MVAR, all with sPOD):
DYN3 (high noise) 0.98,
DYN4 (strong attraction) 0.95,
DYN2 (hypervelocity) 0.91,
DYN6 (variable speed) 0.89,
DYN1 (gentle baseline) 0.81.
1-step teacher-forced R$^2$ was 0.93--0.98 across all, and the POD
ceiling exceeded 0.99.

\textbf{Main findings:}

\begin{itemize}[leftmargin=*,itemsep=2pt]
  \item \textbf{Alignment dominates.}  $\sqrt{\rho}$ + simplex help
    1--3\% in R$^2$ but become negligible once shift alignment is on.
    Alignment alone raises R$^2$ from $\sim$0.5--0.7 to $\sim$0.95+.

  \item \textbf{R$^2$ degrades over time} depending on regime.
    Clustering regimes maintain high R$^2$ past 200s; after sPOD the
    quotient-space dynamics are nearly Markovian.

  \item \textbf{Strong repulsion} (no attraction): lower performance
    because shape dynamics are more complex.

  \item \textbf{Singular value decay:} $d = 19$ captures $>$99.9\%
    of variance with sPOD vs.\ $\sim$96\% without.
\end{itemize}

\subsection*{LSTM Fix and Regime Testing}

I fixed the LSTM training code: wrong config keys
(\texttt{hidden\_size} $\to$ \texttt{hidden\_units},
\texttt{epochs} $\to$ \texttt{max\_epochs}) and missing parameters
(dropout, gradient clipping, layer norm).  The LSTM now runs across
all regimes.  Both LSTM and MVAR produce comparable results in
clustering regimes.

\subsection*{Proposed Next Steps}

\begin{enumerate}[leftmargin=*,itemsep=2pt]
  \item \textbf{Systematic MVAR vs.\ LSTM comparison.}
    Run both models across DYN1--6 plus milling, double mill, swarming,
    and escape (regimes from the Bhaskar topology paper and the
    d'Orsogna model) under identical conditions (same horizon, POD
    threshold, training size, particle count, hyperparameters).

  \item \textbf{Per-regime analysis.}  For each regime: POD vs.\ sPOD
    eigenvalue decay, forecast R$^2$ over time (MVAR vs.\ LSTM), true
    vs.\ predicted density snapshots, and order parameter tracking.

  \item \textbf{Thesis figures.}  Combine SVD decay and phase dynamics
    with empirical results across all regimes.

  \item \textbf{If time allows, alternative sPOD methods.}
    (a) Kevrekidis eigenvalue alignment via the Hermitian
    pairwise-connection matrix.
    (b) Reiss et al.\ multi-frame sPOD for multiple clusters.

  \item \textbf{WSINDy.}  Fit WSINDy to all regimes and compare.
\end{enumerate}

\clearpage
\subsection*{Figures}

\begin{figure}[h!]
\centering
\includegraphics[width=0.7\textwidth]{../artifacts/thesis_figures/svd_decay_raw_vs_aligned.pdf}
\caption{Singular value decay: standard POD vs.\ sPOD.  With sPOD,
  19 modes capture $>$99.9\% of energy.}
\label{fig:svd_raw_vs_aligned}
\end{figure}

\begin{figure}[h!]
\centering
\includegraphics[width=0.8\textwidth]{../artifacts/thesis_figures/dyn_suite_svd_comparison.pdf}
\caption{sPOD singular value spectra across the DYN suite.  Spectral
  knee at modes 3--5 for most regimes; DYN6 (variable speed) decays
  slower.}
\label{fig:dyn_svd}
\end{figure}

\begin{figure}[h!]
\centering
\includegraphics[width=0.8\textwidth]{../artifacts/thesis_figures/r2_vs_horizon_comparison.pdf}
\caption{Forecast R$^2$ vs.\ prediction horizon.  I tested each regime
  at different horizons; the final pipeline will use the same horizon
  for all.  Clustering regimes stay high past 100s; non-clustering
  regimes degrade faster.}
\label{fig:r2_horizon}
\end{figure}

\begin{figure}[h!]
\centering
\begin{subfigure}[t]{0.48\textwidth}
  \includegraphics[width=\textwidth]{../artifacts/thesis_figures/kde_alignment_check_DYN1_gentle_v2.png}
  \caption{DYN1 (gentle Vicsek)}
\end{subfigure}
\hfill
\begin{subfigure}[t]{0.48\textwidth}
  \includegraphics[width=\textwidth]{../artifacts/thesis_figures/kde_alignment_check_DYN6_varspeed_v2.png}
  \caption{DYN6 (variable speed)}
\end{subfigure}
\caption{KDE alignment check.  Left column: raw density.  Right column:
  after shift alignment.}
\label{fig:kde_align}
\end{figure}

\begin{figure}[h!]
\centering
\includegraphics[width=\textwidth]{../artifacts/thesis_figures/phase_dynamics_analysis.pdf}
\caption{Phase dynamics of the shift sequence $\boldsymbol{\Delta}(t)$.
  (a) Shift trajectories.
  (b) PSD: energy concentrated at low frequencies.
  (c) AR($p$) predictability: AR(1) R$^2 > 0.95$, AR(5) $\sim$0.999.
  (d) Phase variance across experiments.
  (e) Autocorrelation: slow decay confirms persistence.}
\label{fig:phase}
\end{figure}

\end{document}
